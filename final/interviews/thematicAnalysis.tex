\subsection{Interview 1}
\label {sec:thematicAnalysisInterview1}
\begin{enumerate}  
    \item Many systems in the workplace are in "legacy mode".
    \item There is a knowledge gap due to the pandemic and staff being moved around.
    \item Documentation is lacking and not helpful in troubleshooting.
    \item When things go wrong, it takes a lot of time to diagnose and fix them.
    \item There is a reliance on poorly documented "clever solutions".
    \item There needs to be more flexibility and adaptability in the workplace.
    \item There are many outdated servers that need to be replaced.
    \item There is a lot of work to do in terms of reorganizing and replatforming systems.
    \item There are multiple single point failures in the workplace.
    \item There are unique challenges in balancing outdated and modern technology.
    \item Third parties want to assess the data centre to see if it can be integrated into a larger facility. The school is hesitant about this because they want to maintain their current system and way of working.
    \item Staff members have already expressed resistance to the idea of integrating into a larger facility.
    \item Interviewee 1 is trying to get the CVL cluster into a state where it is clearly managed to the best practices
    \item Interviewee 1 and Interviewer discuss the importance of well-labelled and well-thought-out cabling and documenting.
    \item They mention the need for a dynamic system like Netbox to help with documentation.
    \item They discuss the potential of allowing students access to the server room to learn about server management.
    \item Interviewee 1 and Interviewer discuss the lack of historical data and the need for a journaling/auditing feature once they have a clean system.
    \item It would be beneficial to capture a document that outlines a system's specifications, owner, and status, as well as the Ansible config used to build it.
    \item A third party recommends killing and rebuilding virtual machines rather than upgrading them through Yum.
    \item Changes made to a system should be documented in a journal to make troubleshooting easier in the future.
    \item An inventory system for tracking physical devices (e.g., servers, switches, cables) is needed, and it should use unique identifiers for each connection.
    \item QR codes should be placed on the front and back of servers to make identifying them easier.
    \item Sockets on the back of servers are poorly annotated and need some kind of identifier.
    \item A stretch goal is to develop an AR feature for visualizing the inventory system.
    \item The project should be designed with future improvements in mind, as new people will come and go and may need to add to or modify the system.
    \item Interviewee 1 has seen AR apps for data centres, but they look awful, and the students could do something better.
    \item AI can do better than a human being in certain tasks, such as pathology.
    \item There is a need for a simple dial to show cooling capacity in the data centre.
    \item Power cables and distribution boards need to be documented and visualised.
    \item Netbox is limited in terms of adding connections and searching, and its hierarchical structure is not ideal for adding devices.
    \item Sunbird is too much for their needs.
    \item There is a need for an app to navigate the data centre, like Pathfinder Mobile, to identify which devices are plugged in and their connections.
    \item Interviewee 1 likes the hierarchical view in the master plan, especially in the rack view.
    \item The hierarchical view is useful for visual navigation and interaction in the academic context.
    \item Interviewee 1 acknowledges the importance of having accurate and up-to-date documentation of network points and devices in different rooms and locations within the building.
    \item Interviewee 1 suggests combining the hierarchical view with documenting network cabinets or comms rooms in the future.
    \item Interviewee 1 mentions the idea of using an app to capture live socket data to be able to trace back connections.
    \item Interviewee 1 suggests adding the tracing feature after Interviewer’s dissertation.
\end{enumerate}

\subsection{Interview 2}
\label {sec:thematicAnalysisInterview2}
\begin{enumerate}
    \item Interviewee 2 thinks that the current state of the server has massively improved and that the future of it is great.
    \item The lighting improvements has made a massive difference, and power distribution has been sorted out properly without tripping.
    \item Documenting and organizing the layout of cables is important, and a focus view on Netbox could be useful.
    \item Replicating the journaling features of networks on an institute level could be beneficial.
    \item End of life and warranty information should be easily accessible in an alert.
    \item QR codes could be implemented to simplify the process of adding connections to assets.
    \item The idea of scanning cables to see the trace of connections is good.
    \item Too much information on one device can be overwhelming.
    \item AR representation of the server room and its devices would be ideal.
    \item Sunbird is too expensive and has too much information.
    \item Netbox is easier to use and visually better than Sunbird.
    \item A visual representation, such as AR, is easier to relate to than data.
    \item Interviewee 2 likes the visual arrangement of Netbox and thinks it is feature-rich.
    \item Having all the information in one place is helpful.
    \item Knowing all the information beforehand is brilliant for diagnosing issues without a site visit.
    \item The app could almost tell you what was going wrong if it's DNS DHCP issue.
    \item People move rooms without telling you, so having an inventory is useful.
    \item It would be good to have a history of changes to data.
    \item The set tags that the university uses could be used for each PC asset.
    \item A third party had all the information initially before all the walls in the building moved, and he could filter it down to Mac address and data point per switch.
\end{enumerate}

\subsection{Interview 3}
\label {sec:thematicAnalysisInterview3}
\begin{enumerate}
    \item The lack of digital or physical records for devices in the server room is the biggest problem.
    \item Modifications or upgrades of the servers are difficult due to the lack of documentation, especially for cables and console panels.
    \item Sometimes it's hard to access the servers or devices, and most of them don't have a warranty anymore.
    \item The error information needs to be gathered, and it's important to have well-considered ways to present data and record it.
    \item The server room was not designed for the purpose, and it's not necessary to have a complex system or a massive software for recording data.
    \item The app needs to have an easy and human-friendly way to add information, and form applications would be helpful.
    \item Providing some basic tools like ping and console access remotely would be helpful.
    \item Safety risks need to be considered for providing wireless signal or access to VLAN, and using a proxy could be a solution.
    \item It would be interesting to simulate live sessions with a proxy, and existing solutions should be explored.
    \item Interviewee 3 suggests that only certain people should be able to access the application.
    \item Interviewee 3 proposes adding labels to devices to make it easier to locate them and scan with the same method as cables.
    \item Interviewee 3 suggests improving the process of adding a connection by allowing users to scan device A, scan device B, and choose the interface.
    \item Interviewee 3 agrees that a view connection feature that allows users to scan the cable and search by device would be helpful.
    \item Interviewee 3 suggests that a network topology feature would be useful for viewing the entire topology of the room.
    \item Interviewee 3 states that adding one interface at a time is fine for normal servers but suggests adding an option to create multiple interfaces in bulk for devices like switches.
    \item Interviewee 3 recommends simplifying the process of showing devices by just scanning the device label and including information about interface connections.
    \item There is a need to include information about hardware settings, such as CVL, for some servers.
    \item There should be a way to view devices across different racks and filter them efficiently.
    \item Adding new devices and fields is slow and needs improvement.
    \item Interviewee 3 prefers Netbox over DC Track as it is more lightweight and less overwhelming.
    \item There is a need to add a feature for upgrading or modifying existing devices, such as changing connections.
    \item Interviewee 3 suggests adding quick links to important information, but it should be managed properly to prevent clutter.
    \item Interviewee 3 prefers to use laptop instead of mobile app for filtering and searching assets.
    \item Adding device make and model templates manually is tedious.
    \item Existing tools are not designed for mobile phones and need to be more human-readable.
    \item The viewing experience of Patch Manager on mobile is not great, but the UI looks better on the platform.
\end{enumerate}
