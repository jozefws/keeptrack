\documentclass [11pt,a4paper]{article}
\usepackage[margin=1.2in]{geometry}
\usepackage{graphicx, lscape, url}
\usepackage{wrapfig}
\usepackage{float}
\usepackage{textcomp, gensymb}
\usepackage{subfig}

\setlength
\parindent{0pt}

\begin{document}
 
% Titlepage
\thispagestyle{empty}
\begin{center}
    \centering
% University logo
    \includegraphics[width=0.5\linewidth]{images/nottingham-logo.png}
    \vspace{2cm}
    {\LARGE \\University of Nottingham\par}
    {\Large Department of Computer Science\par}
    \vspace{3cm}
% Project title
    {\Large COMP3003 - Interim Report\par}
    \vspace{0.5cm}
    {\LARGE A Cross-platform Networking Configuration \& Auditing Mobile Application\par}
    \vspace{2.5cm}
% Author Name
    {\Large Jozef W. Sieniawski\par}
    \vspace{.5cm}
    {\small Computer Science BSc. \par}
    {\small 20296126 $|$ psyjs25@nottingham.ac.uk\par}
    \vspace{1cm}
% Supervisor
    {\normalsize Supervisor: Prof. Chris Greenhalgh\par}
    \vspace{5cm}
% Date
    {\Large December 2022}
\end{center}

\pagebreak
\pagenumbering{roman}    

\begin{abstract}
    \noindent
    A unique set of challenges are faced by small and medium companies that host their own server spaces. Although these devices are typically business-critical, they are typically squeezed into encumbered spaces, with lacklustre lighting and limited access. Further, and critically, this makes the maintaining and documenting of these devices inherently more difficult. Current alternative solutions fail to focus on in-situ use and are typically bloated with features for large data centres. This project will investigate the needs and requirements that can solve these challenges, with a focus on Human-Computer Interaction. From these designs, this project then implements a cross-platform mobile application that satisfies these needs. The following document is the interim report showing progress and methodology so far.

    % Some of the findings. Rather than a table of contents.

\end{abstract}

% Table of contents
\pagebreak

\tableofcontents
\pagebreak 
\pagenumbering{arabic}    

% Add spacing between paragraphs
\setlength{\parskip}{2ex}

% Introduction
\section{Introduction}
\label{sec:introduction}
With the work for the Research Support Team in the School of Computer Science at Nottingham, one of the primary responsibilities revolves around server spaces. This involves the installation and configuration of new servers, as well as the maintenance of existing hardware. When faced with the task of migrating servers to a new location, the problem of understanding the configuration of the existing servers arose. There lacked a consistent documentation format that could be used to replicate the configuration of a server in a new location.

With a larger project in mind, where understanding the configuration of many servers was essential, the idea of this project was born; To create a tool that will allow for cable configuration of server hardware to be easily digitised, visualised, queried, and updated.

Whilst alternatives exist, they are typically a segment of a far larger suite of tools, which are usually not necessary for small/medium-sized server spaces. Naturally, cable configurations can be difficult to understand and work with even in these smaller spaces. This brought forward the second aspect of the project, also an investigation into the user experience and interface design of the tool. To ensure that the tool is easy to use and can be understood by a wide range of users.

Further, the project will be open source and will be available for use by the wider community of server administrators. This will allow for the human Computer interaction findings that have been implemented to be used in similar applications. Additionally, the tool will be utilising and integrating with Netbox\cite{Netbox}, an open-source tool for managing network infrastructure. This will act as the backing database for the tool and will allow for the tool to be used in a wider context of server management.

The Research Support Team will act as a prime example of the target audience. The project aims to discover the needs of a wider range of small and medium-sized enterprises (SMEs) that run and maintain their own server spaces. As mentioned previously, there are not many similar alternatives to the project as these SMEs usually pose a unique server environment. These spaces can be cramped, poorly lit and hard to navigate. Seen below is a picture of the server space in the School of Computer Science at Nottingham (fig. 1), which is an example of the type of environment that the project will be designed for.
\begin{figure}[H]
    \centering
    \includegraphics[width=0.50\linewidth]{images/server_racks.jpg}
    \caption{Server Rack within the school of Computer Science}
    \label{fig:server_rack}
\end{figure}

This is another reason that typical solutions cannot usually apply to these spaces, as they are often designed for large data centres, where laptops can be used easily. A tool that can be used on a mobile device in these less-than-ideal conditions is something worthy of investigation. As mentioned the server space in Fig. \ref{fig:server_rack} is a good example of this. Before upgrades, the space was poorly lit and is, still, relatively cramped. It’s not particularly feasible to use a laptop in this space comfortably, which most solutions rely on due to cluttered UI. The current layout of servers and hardware makes tracking cables completely impracticable and a mobile application would be a great aid in the solution to this problem. 

\begin{figure}[H]
    \centering
    \includegraphics[width=0.55\linewidth]{images/server_racks_clean.jpg}
    \caption{More ideal and realistic server space}
    \label{fig:ideal_server_space}
\end{figure}

Comparatively, the server space dedicated to networking has been managed to a more ideal state (fig. \ref{fig:ideal_server_space}). Whilst not perfect, comparing that of fig. \ref{fig:server_rack}, it is far more manageable. This is the aim of the project, to create a tool that can be used in these less-than-ideal conditions to allow for scenarios like the one in fig. \ref{fig:server_rack} to be said, more manageable.

\subsection{Aims \& Objectives}
\label{sec:objectives}
The projects aims with their respective objectives.
\begin{enumerate} 
    

    \item[A1] To create a tool that will allow for cable configuration of server hardware to be
    easily digitised, visualised, queried and updated. 
        
    - To achieve this, the tool will integrate with Netbox, an open-source tool for managing network infrastructure. This will act as the backing database for the tool and will allow for the tool to be used in a wider context of server management.

        - The tool will use Netbox's API to query and update the database whilst using intuitive UI to allow for easy use.
        
        - To achieve visualisations, the tool will illustrate the cable connections between devices, showing data clearly and concisely.
        \pagebreak
    \item[A2] To create an in situ cross-platform mobile app that can be utilised in restricted
    spaces.    

    - To achieve this, the tool will be cross-platform and will be able to be used on any mobile device supported by the Flutter framework.
    
    - The tool will be designed with a focus on Human-Computer Interaction, to ensure that the tool is easy to use and can be understood by a wide range of users.

    \item[A3] To create an app that can interact with other open-source software easily
    
        - The tool will be open source, so modifications can be made to the tool to allow for it to be used in other contexts.
        
        - The tool will be built around modifiable models that can be changed easily for other software or a custom build backend.

\end{enumerate}

\subsection{Background}
\label{sec:background}

There are currently no popular mobile tools for this use case, but there exists a need for a solution. Whilst it can be appreciated that in large data centre applications, it is likely that a mobile application like this might not have the same functionality as a desktop application, which is likely more desirable. But with personal experiences working for the Research Support Team a tool of this nature would be extremely useful. Further, with the Open Source nature of Netbox, the accepted DCIM software for RST, a mobile application would be a perfect companion. Further then, the tool could then be released to the wider community of server administrators to be used with their instances of Netbox.

\section{Related Work}

In this section, three aspects of related work are discussed; applications similar to the project, related human-computer interaction research, and finally, papers that are related to the project technically.

\subsection{Application and Product Reviews}
\label{sec:app_reviews}

Following research on open-source and paid-for cable management/DCIM software, it was found there are limited options that allow for the trial version without a legitimate business interest. This restricted the freedom of choice in the related application that could be investigated. This aside, there follows three different software packages that are relevant to the project. Including; Sunbird DCIM\cite{Sunbird}, a paid-for solution with a trial accessible publicly. Next, Pathfinder Mobile \cite{Pathfinder}, a mobile counterpart to the enterprise "Pathfinder" package, was the only high-quality mobile solution that was discovered. Finally, Netbox \cite{Netbox}, the open-source DCIM software that the project will be built around. These three packages were shown to three individuals within the Research Support Team and their feedback was recorded. Each software was shown in mobile view. 

\subsubsection{Sunbird DCIM}
\label{sec:sunbird}
Sunbird DCIM is a feature-full Data Centre Infrastructure Management (DCIM) Package with a significant list of components. Their client list includes the Paddypower Betfair, eBay \& COMCAST \cite{Sunbird-we-know-data-centres}. When trialling dcTrack their DCIM software - The immediate impression is that it is feature-rich, including monitoring environment, security, cooling as well as having asset and connectivity tracking. Whilst the server spaces within Computer science at UoN could benefit from a tool like this, its implementation and management would be strenuous. The tool is designed for large data centres and would be overkill for the server spaces within the school. For example, in fig \ref{fig:sunbird_dcTrack}, the space is approx 10x larger than that at Computer Science. 

\begin{figure}[H]
    \centering
    \includegraphics[width=0.8\linewidth]{images/sunbirddcim.png}
    \caption{Sunbird DCIM - dcTrack}
    \label{fig:sunbird_dcTrack}
\end{figure}

Though, some of its features could be useful in the implementation of the project. For example, the visualisation of data centres via a 3D model could be an interesting feature to implement in the project but might be out of scope. On the other hand, the search for assets and connections that Sunbird has, while thorough, is not as intuitive as the project aims to be. Further, Sunbird intends to serve clients that might have thousands of assets, not quite hundreds like Computer Science has, so the requirements of searching and filtering are different. Though filtering by an extensive set of categories is a useful feature and could be implemented in the project, but in a more intuitive manner.

\subsubsection{Pathfinder Mobile}
\label{sec:pathfinder}

Pathfinder Mobile is a mobile component to the complete Pathfinder package, it is designed to be used in conjunction with the full software and used as an "anytime and anywhere" \cite{PathfinderMobile} tool. It allows existing users of Pathfinder to access data remotely and in situ, with an intriguing focus on "work orders". These are created at a workstation, i.e. a laptop, which then synchronizes the work orders with the mobile app. From here the work orders can be executed on-site by using "graphical instructions support"\cite{Pathfinder}. Finally, on completion, all changes are uploaded to the pathfinder client. See below for screenshots from the Pathfinder Mobile app.

\begin{figure}[H]%
    \centering
    \subfloat[\centering Side Panel]{{\includegraphics[width=4cm]{images/pathfinder_overview.png} }}%
    \qquad
    \subfloat[\centering Network Trace]{{\includegraphics[width=4cm]{images/pathfinder_trace.png} }}%
    \caption{Screenshots of the Pathfinder Mobile App}%
    \label{fig:pathfinder_screenshots}%
\end{figure}

A similar environment interaction would work well for this project, with more complex modifications being completed/generated on a desktop client on the school's Netbox instance, i.e. Templating. Then once completed, the data entry for these templates can be completed on the mobile app, in the server space - with the app synchronising with Netbox via its API. 

The pathfinder app also allows for quick access to networking information, where users can complete the tracing of connections. Whilst this already was a core intention of the project, Pathfinders' method to visualise this is intuitive and similar to that which was discussed in the first Current State Analysis interview. These interviews are discussed more in detail in section \ref{sec:current_state_analysis}. But, notably, an interviewee mentioned a good method to do a visualisation is to list devices in a scrollable view - with the connections between them being drawn on the screen, along with device and interface information displayed as well. This is similar to the method used by Pathfinder, with; device name, location, interface name, type and cable type being displayed. 

One aspect of Pathfinder that could be improved upon is the search for assets. Whilst they describe their search as being "text-rich". It seems to be less intuitive, and does a search based on every field, meaning that results for a simple search can be overwhelming. This is something that the project aims to improve upon, with a more intuitive search method. Further, their search by barcode scanner only simply enters the code into the search bar, not necessarily a bad method, but not as intuitive as it could be. 

Overall, Pathfinders mobile app will be a good reference point for the project. It has a similar focus to the project and has a similar method of visualising data. Though it is a companion of a larger software package that exceeds the current requirements of the school, it has features that can be referred to and implemented in the project.

\subsubsection{Netbox}
\label{sec:netbox}
Whilst searching for a DCIM solution for the School of Computer Science - Netbox became a clear choice. It has a solid feature core, it's open source, self 'hostable' and can help solve the problems that the school is facing. As discovered during the Current State Analysis and personal experiences, the server rooms within CS are documented poorly due to a lack of consistent historical data. Most of the information is stored in the memory of long-gone staff or out-of-date spreadsheets. Netbox is to act as a "source of truth"\cite{Netbox}. More information on Netbox is included in the Implementation section \ref{sec:development}. 


\subsection{Papers focused on HCI and User Experience}
\label{sec:HCI}
There is a lack of significant research into areas of Human-Computer Interaction and User Experience in the context of server management. But there are a few papers that are relevant to the project. The first paper is a write-up by Yin et al. \cite{cloud3dview} regarding their demonstration of Cloud3DView at SIGCOMM '13. Cloud3DView is an interactive 3D visualisation tool used for Data Centers which uses FPS gamification to allow users to monitor situations and control data from a user-friendly interface. Cloud3DView also included a focus on "cutting-edge HCI technologies" \cite{cloud3dview}. Whilst the paper itself doesn't go into the reasoning of why certain HCI choices were made - It shows a good example of how a mobile device can be used to interact with a data centre and can be used to improve efficiency and user experience. Whilst the focus of this project isn't to be a 3D visualisation tool, one aspect of Cloud3DView is the ability to view data about the devices in a 3D environment. Whilst, a 3D visualisation is out of scope for KeepTrack - It is possible that an augmented reality view could be implemented, where users can view data/connections concerning the devices via a camera on their mobile device. This was a feature suggested by all interviewees during the Current State Analysis.

Further, there was a focus on using mobile devices to be used for the visualisation but didn't mention the use of a mobile device for data entry. With the nature of the data being entered in the context of data centres, a mobile format will be more intuitive/accessible than a desktop format in some cases. An important benefit of using mobile devices for data entry is the ability to use multiple interface types to enter data. For example, a user can enter data via an alphanumeric keyboard, numerical keypad, calendar picker, lists etc. A comparative analysis into different data entry design patterns investigated 3 different patterns for 7 different data types for a set of 9 tasks testing different patterns \cite{myka2019comparative}. For this project, the most relevant data types are; dates, small numbers, single-choice lists and text. The investigation completed a set of usability tests to determine which pattern was the most effective, with different patterns benefiting different tasks/data types better than others. The results from this study will be used to inform the design of the interface for the project.

On a similar note, a study into the structure of data entry on mobile devices noted that existing interfaces "interfere with user input" and force "complex interactions to enter simple information" \cite{van2007gui}. This is especially apparent in some cases of the Netbox interface e.g., as mentioned previously, adding cable connections. Whilst, naturally, there are limitations as to how streamlined an interface can be on a desktop-based web application, there is more room for improvement on a mobile application.
\pagebreak

Kleek et als \cite{van2007gui} study also focused on the structure of data entry on mobile devices, its importance, but also the importance of not creating interfaces that deter users from entering data. Highlighting a fine balance between the two. One of the sources of problems for the school is the lack of data ever being entered anywhere. If the project has an interface that dissuades users or is perceived as too tedious then it will not solve a key problem. This project will take upon the findings of this study to ensure that the interface is intuitive and easy to use. For example, an autocomplete search bar for assets and replacing text fields with scanners or other input methods. 

\subsection{Papers focused on Technical aspects of the project}
\label{sec:technical}

One of the key parts that the project aims to improve is the searchability of assets, i.e. finding the asset desired quickly and accurately through many means. The current method of asset identification is inconsistent with different naming conventions, ID codes, labels, no labels, hostnames etc. The Centre for the Protection of National Infrastructure (CPNI) has a paper on the importance of asset management, specifying that, "all organisational assets and systems that are necessary for the delivery of effective operations or are of specific organisational value, should be identified." \cite{cpni}.

Not only is it important to identify assets whilst they are in use, but also for documenting any changes, journal entries, ownership, location etc. This project also aims to improve upon this, with a focus on the identification of assets being homogenous and consistent.

In this project, assets are considered to be both physical hardware, such as servers, switches, KVMs etc. but also cabling, such as ethernet, fibre, power etc. To identify each of these assets uniquely it was decided that an encoded label would be best. A comparative study into barcodes, QR-codes and RFID systems in the library environment looked into the respective advantages and disadvantages of each \cite{lotlikar2013comparative}. Going by these attributes, it is more sensible to use Barcodes or QR codes for the project. With RFID tags being more expensive, tag collisions can occur when many are in close proximity\cite{lotlikar2013comparative} - Like that of a server room. Further, a specialised reader is likely needed and so another handheld device would need to accompany the app.

Additionally, a paper written on the review of QR codes highlights that QR codes can store more data in the same area, have data redundancy, are faster to scan and are "readable from any direction in 360\degree"\cite{mishra2017review}.  Whilst these benefits seem to make it the clear choice over barcodes, QR codes contain data in both dimensions - Whilst this is a benefit in most use cases, it is likely to pose a problem when scanning codes stuck to ethernet cables.

Where, barcodes contain data in one dimension which is then repeated vertically, allowing for their placement to be more dynamic, e.g. on a cable. This is clearly shown in figure 3 of Mishra et al., fig. \ref{fig:barcode}. So, for this project the use of barcodes for cables, and then QR codes for other assets will be used. Especially as the information stored for cables will be a random short number assigned to each cable, suitable for barcodes. On the other hand, with assets, e.g. servers, the label can be more specific, so a QR code will be used. E.g. contain a URL that links to the asset in the Netbox database.

\begin{figure}[H]
\centering
\includegraphics[width=.75\textwidth]{images/barcode_mishra.png}
\caption{Barcode vs QR code}
\label{fig:barcode}
\end{figure}

Whilst an Augmented Reality visualisation feature is dependent on time and the success of other elements. It is still important to consider related work with these features, especially as it could be a feature added on. Whilst already investigating Cloud3DView \cite{cloud3dview}, which uses 3D visualisation, a more apt implementation would be by using the QR codes placed on assets to aid in visualising data. This is very similar to the work outlined in the paper, "Applying QR code in augmented reality applications" \cite{applyingQR}. This paper displays how the elements of the QR code can be used to position elements in the AR scene and simultaneously generate information embedded in the QR code. Should this be a feature that can be implemented, it is likely the project will follow a similar method as to what Kan et al. set out \cite{applyingQR}. With the existing desire to implement a multi-format code scanner, i.e. both barcode and QR, this would be a good addition to the project. Further, there already exists a cross-platform plugin for Augmented Reality in Flutter which can be utilised \cite{ar_flutter}.

\section{Description of the Work}
\label{sec:work}
This project aims to create a mobile application that will allow users to add, view and manage assets and cables in server rooms. With the specific application to Small/Medium enterprises. The application will be built using Flutter, a cross-platform framework which will allow for the app to be used on both Android and iOS devices. The application will be built upon Netbox DCIM \cite{Netbox} and their RESTful API \cite{NetboxAPI}. 

The project aims to build upon this API to create a more user-friendly and intuitive interface for server administrators to use. With this in mind, a second key aspect is investigating the best interface for data entry in this context. This will be done by interviewing and observing the system administrators at the school using iterative prototypes of increasing fidelity.

\subsection{Project Stretch Goals}
\label{sec:stretchgoals}
Following the interviews during the current state analysis and the investigation of similar work, the further goal to implement an augmented reality visualisation feature is desirable. This feature will allow users to visualise asset information and networking details in a highly interactive way. 

Desirable features that could be implemented depending on time constraints in priority order: 
\begin{itemize}
\item Journals for assets, allowing for notes to be added to assets, e.g. when a new component is added, or when a component is removed.

\item Auditing of assets, allowing for the user to scan an asset and then be presented with a form to fill out, e.g. the asset is in good condition, or the asset is faulty. 

\item In-App SSH terminal, allowing for the user to connect to a device and run commands, e.g. to check the status of a service.

\item Augmented reality to visualise asset information and networking details.
\end{itemize}

\section{Methodology}
\label{sec:methodology}
With all of the aims and aspects of the projects in mind, it was decided to use a participatory design approach, with iterative prototypes of increasing fidelity. Then, with user evaluation and feedback, the final product will be created. This will be done by following the steps outlined in the following list:

\begin{enumerate}
    \item Create low-fidelity prototypes
    \item Current state analysis and initial interviews
    \item Create higher fidelity interactive prototypes
    \item User evaluation and feedback
    \item Minimal Viable Prototype with user feedback
    \item Create final product
\end{enumerate}

Whilst deciding on a cross-platform framework to build this project on, Flutter become the obvious choice. Not only with previous experiences working with it, but also with the fact it has become the most popular cross-platform framework \cite{JetBrainsFlutter}. Further, with personal work in aiding the development of Labmonitor\cite{labmonitor}, there is a familiarity with the code scanning package that will also be used in this project\cite{barcodeScannerPlugin}.

\subsection{Current State Analysis and Initial Interviews}
\label{sec:current_state_analysis}
To begin with, using personal experience, knowledge and  understanding of the requirements, a set of low-fidelity prototypes were made - primarily to establish an initial set of features. A set of interviews were conducted with the system administrators (RST) to extend upon original experiences.  Within these interviews, the current state of the system was discussed, to understand the current workflow and the problems that are faced.
\pagebreak

This was done by asking broad questions such as:

\begin{itemize}
    \item What are your opinions on the current state of the server room?
    \item What do you think could be improved?
    \item Do you have any suggestions for the new system?
\end{itemize}

Then, each interviewee was showed the low-fidelity prototype of the application, intending to gather initial thoughts on the proposed layouts and to see if there were any features that they would like to see added/removed. Their responses were recorded and then analysed, to identify any repeating ideas, mainly using thematic analysis \cite{thematicAnal}.

\section{Design and Implementation}
\label{sec:design}
The following section will outline the design and implementation of the project. Specifically; the user interfaces design, the system architecture and the current implementation.

\subsection{User Interface Design}
\label{sec:ui_design}
Following the initial interviews, a set of high-fidelity prototypes are being created. These are currently being created using Figma \cite{figma}, a web-based prototyping tool. Chosen due to extensive prior use and its high reputability, with the UX Design Institute describing it as the best prototyping tool \cite{figmaUX}. With Figma's prototyping features, it is possible to create interactive prototypes, these will then be used to conduct further interviews, but with a more evaluative, task-based, focus. Then based on the feedback, an initial Minimal Viable Product (MVP) will be created. This will be done by following the steps outlined in the next section. Then an evaluation will be made on the MVP, to inspire any last changes/features in the final design, which then will be used to build upon the MVP to create the final product. See below for the current high-fidelity prototype progress.

\begin{figure}[H]
    \centering
    \includegraphics[width=1\textwidth]{images/figma_prototype.png}
    \caption{Figma Prototype}
    \label{fig:figma}
\end{figure}


The interface will be based on the Material Design, which is described as "an adaptable system of guidelines, components, and tools that support the best practices of user interface design."\cite{materialDesign}. It has several User Experience (UX) principles at its core (such as accessibility) that will create a user-friendly interface. Material is widely used and will give users a sense of familiarity, especially icons and GUI elements. This similarity will increase learnability and reduce the user's cognitive load.

\subsection{Development and System Architecture}
\label{sec:development} 
For the development of the app, as mentioned and reasoned in section \ref{sec:technical}, Flutter will be used to create the app. The following diagram describes how the app will interact with the Netbox Instance using its REST API. These calls are processed through NGINX which is acting as a reverse proxy.

\begin{figure}[H]
    \centering
    \includegraphics[width=0.6\textwidth]{images/top-level-archi.png}
    \caption{System Architecture}
    \label{fig:architecture}
\end{figure}

The app will use Netbox's inbuilt Django user authentication system, which allows for tokens to be created via an API call. This token will then be used to authenticate the user to fetch data and make changes to the Netbox instance, returning any errors should a permission problem arise. For stretch goals, should the app feature more information or processing - The requests will still go through the same NGINX reverse proxy but to a different API endpoint - Though open to change, a Python-based Flask Server will be used. 

The app will be structured using a Data Access Object (DAO) pattern\cite{dao}, to separate the data access from the logic. In a way, replicating the data objects within Netbox itself. This will allow for the app to be easily extended and modified in the future. 

Further, there is a separation between views, API calls and providers. This gives a more modular structure and provides more maintainable code, especially as requirements change, or Netbox updates/changes. Additionally, in the spirit of Open Source Software, the code will be stored on GitHub \cite{keeptrackgithub} and will utilise environment variables to store sensitive information so that other developers can easily implement their instance of the app.

\subsection{Implementation}
\label{sec:implementation}
The implementation has focused on the development of the core logical structure of the app and the creation of a foundation for the UI to test functionality. The current progress allows a user to sign into the app using their Netbox credentials. This information is then stored securely using the Flutter Secure Storage plugin \cite{securestorage}. The current UI allows the user to add a connection between two devices. More specifically, it allows searching for devices via a searchable dropdown, which then dynamically loads its interfaces, with a tab view that allows the user to select the interface on each device. Further, the current implementation has a working integration of the barcode/QR code scanner plugin \cite{barcodeScannerPlugin}. See below for screenshots of the current UI, fig. \ref{fig:currentUIScreenshots}. 

\begin{figure}[H]%
    \centering
    \subfloat[\centering Login Page]{{\includegraphics[width=2.5cm]{images/login.png} }}%
    \qquad
    \subfloat[\centering Add Connection Page]{{\includegraphics[width=2.5cm]{images/addcon.png} }}%
    \qquad
    \subfloat[\centering Device Search]{{\includegraphics[width=2.5cm]
    {images/search.png} }}%
    \caption{Screenshots of the current Keeptrack UI}%
    \label{fig:currentUIScreenshots}%
\end{figure}


\section{Progress}
\label{sec:progress}
The following section will outline the progress made on the project, and the challenges faced. Further, it describes the future work plan and adjustments to milestones. 
\subsection{Projection Management}
\label{sec:project_management}

Whilst the project is still in its early stages, there has been some good progress made, though less so than initially planned. This is due to several factors. Firstly, I hadn't assigned enough weekly time to work on the project, which has led to slower progress overall. Secondly, I hadn't accounted for interviewee availability, which led to a delay in the interviews and therefore the creation of the high-fidelity prototypes. However, I have been able to make progress on the technical side of the project, which has allowed me to create a foundation for the UI and the core logic of the app. I am satisfied with the technical progress made and I believe that with the core foundations now in place, technical progress can be completed more quickly.

\pagebreak

I also didn't expect that the literature review would take as long as it did, or more specifically for this project - Finding similar applications along with technical and HCI-based papers. Finding applications that had easily accessible trials posed a harder challenge than I had anticipated. But I believe that the similar work I have found has given me a better insight into a mixture of HCI and technical aspects of the project. Though, I struggled to find HCI-focused papers that were related to the scenario posed, e.g. encumbered use of a mobile device. 

On a positive note, the interviews brought up some interesting points that I hadn't considered before. For example, an inbuilt SSH terminal, and augmented reality features, but also increasing the scope of the project to include other rooms. Whilst some of the ideas risen can be achieved easily, others will require more time and research. However, I believe that the interviews have been a success and have given me a better understanding of the stakeholders and their needs.

With the delay and change in some aspects of the project, I have had to re-evaluate aspects of the new work plan. Shifting milestones along and re-assigning tasks to more/less work. I believe that in the work plan I hadn't accounted for a large enough research period, i.e. finding similar work, but now that this has been completed, I can focus on getting on with the technical work.

Comparing the initial Gantt chart, seen in appendix \ref{sec:init_gantt_charts}, to the updated Gantt chart, appendix \ref{sec:updated_gantt_charts}, it has become apparent that I have had to shift the milestones along. But this mainly filled most of the run-over time I had assigned in the initial plan. But even with these delays, I am confident that I can complete the project within the time frame given.

During weekly meetings with the project supervisor, we discussed the progress made and the work that needs to be completed. To keep track of the discussions, a document was maintained. Each meeting I would write up a summary of topics I wanted to discuss, and then during the meeting add notes to each section. Then following the meeting, I would review a Trello board (See appendix \ref{sec:trello_board}) to see what tasks I had completed and what tasks I needed to complete. Though, initially, I didn't assign dates to each task this was changed quickly after failing to achieve tasks in coordination with the work plan. Additionally, each task is also colour coded based on work type. 

\subsection{Contributions and Reflections}
\label{sec:contributions}
The primary contributions I have made to the project so far are the literature review, interviews and technical work done to date. From this work, I'll be able to create a set of high-fidelity prototypes and start creating a Minimal Viable Product (MVP) soon after. Although, I am aware that I have somewhat fallen behind on the project. I estimate this to only be a week or two behind, but with the reshuffling of the work plan, and the time set aside for run-over, I am confident that I can complete the project within the time frame given.

Overall I am happy with the progress made so far. Initially, whilst I had a good idea of what the app needed to be, following the work done I have a better understanding of what will make the app useful. Additionally, I have also learnt that there is a significant lacking in the market for a product like this. 


Notably, I think it is important that I reach out to external stakeholders, outside of the university, to get a better understanding of the market and what is needed. It is possible that more ideas will arise from this and that the project will not be limited to what RST needs. 

I think that with the next steps of the project it is important that I establish a more constant workflow, rather than the bursts of work I have carried out so far. I think that this will create more iterative progress and can prompt more user feedback. Notably, I think that there is a bigger risk of feature creep than I expected and I think I will need to pay more attention to this moving forward.

\subsubsection{Laws, Social, Ethical and Professional Issues}
\label{sec:computer_laws}
As this project intends to be open source under the Mozilla Public License Version 2.0 I foresee no implications for intellectual property. It is currently and will be publicly available via Github \cite{keeptrackgithub}. Although, it is possible the work may be exploited for commercial gain but is dependent on the success of the project and the possibility to be implemented externally.

On the topic of ethics, due to the project involving human participants, I have had to complete the full ethics form, "CS REC 1", and have received approval from relevant parties. Further as set out by the Research ethics guidelines \cite{ethicsguidelines}, I have completed the Data Management Plan (DMP) form. This form outlines how I will store and manage the data collected during the interviews. Each interviewee was given a consent form to sign, outlining the purpose of the interview and the data collection. The interviewees have also been given the option to opt-out of the interview at any point.

Additionally, there are notices regarding the Data Protection Act 2018. The data collected will be stored on a password-protected computer and will be deleted after the project has been completed.

I believe that from a broader perspective, this project will not have significant implications for society. Though, I do believe that it will have a positive impact on the RST team and the wider system administrator community. I believe with the project being open source and with a focus on HCI, it will be accessible to a wide range of people. Further giving DCIM software another platform to be accessed, I believe that will be a positive step towards the future of DCIM software and data centre management.

\pagebreak

% Bibliography
\bibliographystyle{ieeetr}
\bibliography{citation} 

% Appendix
\appendix
\section{Appendix}
\label{sec:appendix}
\subsection{Trello Board}
\label{sec:trello_board}
\begin{figure}[H]
    \centering
    \includegraphics[width=.75\textwidth]{images/trello-board.png}
\end{figure}   

\subsection{Initial Gantt Chart}
\label{sec:init_gantt_charts}
\begin{figure}[H]
    \centering
    \includegraphics[width=0.75\textwidth]{images/keeptrack-gantt-initial.png}
    \label{fig:initialworkplan}
\end{figure}

\subsection{Updated Gantt Chart}
\label{sec:updated_gantt_charts}

\begin{figure}[H]
    \centering
    \includegraphics[width=0.85\textwidth]{images/keeptrack-gantt-interim.png}
    \label{fig:updatedworkplan}
\end{figure}


% Document End
\end{document} 